\documentclass{article}

\title{Sztuczna inteligencja}
\author{Dominik Lau}

\usepackage{blindtext}
\usepackage{amsmath}
\usepackage[utf8]{inputenc}
\usepackage[polish]{babel}
\usepackage[T1]{fontenc}
\usepackage{listings}
\usepackage{color}
\usepackage{amssymb}
\usepackage{esvect}
\usepackage{graphicx}
\usepackage{hyperref}



\definecolor{dkgreen}{rgb}{0,0.6,0}
\definecolor{gray}{rgb}{0.5,0.5,0.5}
\definecolor{mauve}{rgb}{0.58,0,0.82}

\lstset{frame=tb,
  language=Python,
  aboveskip=3mm,
  belowskip=3mm,
  showstringspaces=false,
  columns=flexible,
  basicstyle={\small\ttfamily},
  numbers=none,
  numberstyle=\tiny\color{gray},
  keywordstyle=\color{blue},
  commentstyle=\color{dkgreen},
  stringstyle=\color{mauve},
  breaklines=true,
  breakatwhitespace=true,
  tabsize=3
}

\graphicspath{ {./media/} }
\begin{document}
\maketitle
\section{Wstęp}
Celem projektu było zaimplementowanie i przeanalizowanie dwóch
algorytmów interpolacji na wybranych profilach wysokościowych - metody wykorzystującej wielomian Lagrange oraz metody z funkcjami sklejanymi trzeciego stopnia.
Do implementacji wykorzystano język $Python$ oraz biblioteki
$matplotib$, $pandas$.
\section{Teoria}
W obu przypadkach zakładamy, że mamy pewien zestaw $n+1$ punktów
\begin{gather*}
	(x_0, y_0) \\
	(x_1, y_1) \\
	...\\
	(x_n, y_n)
\end{gather*}
i chcemy znaleźć taką funkcję $F(x)$, że
\begin{gather*}
	\forall_{i= 0..n}  F(x_i) = y_i
\end{gather*}
dobrze określającą, jakie wartości przyjmują $y$ w punktach $x \notin \{x_0, ..., x_n\}$
\subsection{Metoda Lagrange}
W metodzie tej funkcja $F$ ma postać
\begin{gather*}
	F(x) = \Sigma_{i=0}^{n} y_i \phi_i(x)
\end{gather*}
gdzie 
\begin{gather*}
	\phi_i(x) = \Pi_{j=0,  j \ne i}^{n+1} \frac{x-x_j}{x_i - x_j}
\end{gather*}
jest \textbf{bazą Lagrange'a}. Metoda ta zwraca takie same wyniki
jak metoda Vandermonde, jednak nie musimy rozwiązywać układu równań liniowych
\subsection{Metoda krzywych sklejanych 3. stopnia}
w tej metodzie funkcja $F$ ma postać
\begin{gather*}
	F(x) = S_i(x); x\in [x_i, x_{i+1}]
\end{gather*}
czyli przedstawiamy ją jako szereg połączonych wielomianów $S_i(x)$ takich, że
\begin{gather*}
	deg (S_i) = 3
\end{gather*}
w celu uzyskania układów równań, z których pozyskamy współczynniki $S_i(x)$ przyjmujemy założenia
\begin{gather*}
	S_i(x_i) = y_i \\
	S_i(x_{i+1}) = y_{i+1} \\
	S_{j-1}'(x_i) = S_{j}'(x_i); x = 1..n-1 \\
	S_{j-1}''(x_i) = S_{j}''(x_i); x = 1..n-1 \\
	S_0''(x_0) = 0 \\
	S_{n-1}'' (x_n) = 0 \\
\end{gather*}
znalezienie wielomianów $S$ sprowadza się do rozwiązania powyższego układu równań, w praktyce można wyprowadzić z powyższych układów
wzory na $b,d$ w zależności od $c$ a następnie rozwiązać cały układ równań dla wektora $\textbf{c} = [c_0, ..., c_{n-1}]$:
\begin{gather*}
	\begin{pmatrix}
	1 & 0  & ...  & 0 & 0 \\
	h_1 & 2(h_1 + h_2) & h_2 & 0  & 0 & ... \\
	0 & h_2 & 2(h_2 + h_3) & h_3 & 0 & ... \\
	... \\
	0  & 0 & 0 & ... & 0 & 1
	\end{pmatrix} 
	\begin{pmatrix}
	c_0 \\
	... \\
	c_{n-1}
	\end{pmatrix}
	= \begin{pmatrix}
		0 \\ 3(\frac{a_2 - a_1}{h_2} - \frac{a_1 - a_0}{h_0}) \\ ... \\ 0
	\end{pmatrix}\\
	d_i = \frac{c_{i+1} - c_i}{3h_i} \\
	b_i = \frac{a_i - a_{i-1}}{h_i} - \frac{h_i}{3}(2c_i + c_{i+1}) \\
\end{gather*}
i właśnie z tak uproszczonego układu skorzystano w implementacji (rozwiązany metodą Jacobiego)
\section{Wybrane profile wysokościowe}
Do analizy wybrano następujące profile wysokościowe
\begin{itemize}
	\item ścieżkę Yoshidy na górę Fuji - jedno duże wzniesienie
	\item trasę Al.  Ujazdowskie-Łazienki-Solec - trasa głównie płaska
	\item trasę wokół centrum Słupska - wiele nagłych (ale niedużych) wzniesień
\end{itemize}
\section{Trasa na górę Fuji}
\section{Trasa w Warszawie}
\section{Trasa wokół centrum Słupska}
\section{Źródła}
\begin{itemize}
	\item \href{https://en.wikipedia.org/w/index.php?title=Spline_%28mathematics%29&oldid=288288033#Algorithm_for_computing_natural_cubic_splines}{Wikipedia-Spline}
	\item \href{https://medium.com/eatpredlove/natural-cubic-splines-implementation-with-python-edf68feb57aa}{Medium-Spline}
\end{itemize}

\end{document}


